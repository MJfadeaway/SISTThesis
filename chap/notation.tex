
\chapter{符号和约定}
\subsection*{本报告对符号的使用做如下约定:}
\begin{tabular}{p{5cm}p{9cm}}
    $\bm{u}$ & 加粗斜体小写字母表示向量\\
    %$\bmu{u} $ &  带下划线的加粗斜体小写字母表示分块向量\\
    $\bm{K}$ & 加粗斜体大写字母表示矩阵\\
    %$\bmu{K}$ & 带下划线的加粗斜体大写字母表示分块矩阵\\
    $a,b,c$ & 斜体小写字母表示数字\\
    $\mathcal{C},\mathcal{I} , \mathcal{F}$ &欧拉书写体表示算子或者$\sigma$-代数
\end{tabular}


\subsection*{本报告中符号的含义:}
\begin{tabular}{p{5cm}p{9cm}}
    $\mu$ & 非负测度 \\
    $\rho$ & 概率密度函数或者权函数\\
    $D$ & 物理空间上的区域\\
    $\Gamma$ & 随机变量空间上的区域\\
    $\langle \cdot, \cdot \rangle_{\rho}$ & 表示以$\rho$为权函数的内积\\
    $\Vert\cdot\Vert_{\rho}$ & 内积$\langle \cdot, \cdot \rangle_{\rho}$的诱导范数\\
    $\mathbb{E}[\cdot]$  & 期望\\
    $\mathbb{V}[\cdot]$ & 方差\\
    $\delta_{i,j}$ & delta 函数\\
    $\otimes$ & Kronecker积\\
    $\mathrm{Cov}_a(\bm{x}_1,\bm{x}_2)$& 随机过程$a(\bm{x},\omega)$的协方差函数\\
    
\end{tabular}
